% Language setting
% Replace `english' with e.g. `spanish' to change the document language
\documentclass[12pt,french]{article}
\usepackage[utf8]{inputenc}
\usepackage[T1]{fontenc}
\usepackage{minted}
\usepackage{babel}
\usepackage{ulem}
\usepackage[babel=true]{microtype}
% Set page size and margins
% Replace `letterpaper' with `a4paper' for UK/EU standard size
\usepackage[letterpaper,top=2cm,bottom=2cm,left=2cm,right=2cm,marginparwidth=1.75cm]{geometry}

% Useful packages
\usepackage{amsmath}
\usepackage{graphicx}
\usepackage[colorlinks=true, allcolors=blue]{hyperref}

\newcommand{\inlinecode}[1]{%
  \texttt{%
    % set flexible interword space
    \setlength{\spaceskip}{0.5em plus 1em minus 0.1em}%
    % add some space with not as much flexibility, but only
    % if some space precedes
    \ifdim\lastskip>0pt \unskip\hspace{0.5em plus 0.5em minus 0.1em}\fi
    #1
  }%
}
\title{Programmation 1}
\author{Azammat Charaf Zadah}
\date{}
\date{}
\begin{document}
\maketitle
\begin{center}
    \textit{Projet 1, pour le 30 Octobre 2022}
\end{center}

\section{Déroulé}
\subsection{Premiers pas}
Au début, j'ai dû revoir ce que signifiait compiler une expression. Si j'ai bien compris, il s'agit d'utiliser un lexeur qui, à partir de l'expression, crée une liste de lexèmes qui sera parsée par - surprise ! - le parseur (ou alors je suis déjà mal pour ce projet). Ensuite, on produit une sortie à partir de l'AST généré (par exemple, au hasard, un programme en langage assembleur calculant l'expression et affichant le résultat).
\subsection{Difficultés}
Problème, si faire une liste de lexème est plutôt accessible, faire un parseur qui respecte les priorités opératoires pour une expression en notation infixe peut être difficile ! (Sans parler de ma maîtrise très approximative du langage assembleur à ce moment-là.) Je commence donc à me dire, avec d'autres de mes camarades, que nous devrions aller voir du côté du GIGN qui propose des postes accessibles sans aucun diplôme, pour démarrer une nouvelle vie si je venais à me faire exclure de l'établissement pour cause de non-validation de l'UE Programmation 1.
\subsection{\texttt{ocamllex} et \texttt{ocamlyacc}}
C'était sans compter l'arrivée \sout{du Messie} de Clément Dumas qui nous révéla l'existence d'\texttt{ocamllex} et d'\texttt{ocamlyacc}, des outils permettant de générer des lexeurs et des parseurs rapidement, avec une gestion aisée de la priorité des opérations. Ils rendent les analyses syntaxique et lexicale à portée de main ! Dès lors, il s'agissait surtout de savoir correctement manipuler l'AST de sortie : vérifier qu'il est bien typé (en gros, ne pas mélanger entiers et flottants), puis le cas échéant, l'utiliser pour générer un programme assembleur qui correspond au calcul de l'expression arithmétique d'entrée et à l'affichage du résultat (merci \texttt{x86\_64.ml}, même si j'ai dû le modifier un peu). Pour cette dernière étape, des révisions du langage assembleur se sont révélées utiles ! J'ai aussi dû me renseigner pour pouvoir manipuler les flottants avec les bons registres et les bonnes opérations en assembleur, avec la documentation et auprès de mes camarades.
\newpage
\section{Résultats}
\subsection{Vue d'ensemble}
Chaque fichier du projet est commenté et précédé d'une en-tête dans laquelle est précisée son utilité (mais leurs noms sont déjà plutôt explicites). Je respecte, je l'espère, les consignes : après avoir exécuté la commande \texttt{make}, on a notamment l'exécutable \texttt{aritha} et ce rapport {\LaTeX}  qui sont générés. La commande \texttt{./aritha expression.exp} exécute le projet sur le fichier en argument pour effectuer la compilation de l'expression arithmetique présente dedans. On affiche  : "\inlinecode{OK, fichier expression.s généré.}" si la compilation a abouti. Sinon, un des messages d'erreur suivants : "\inlinecode{Mauvais typage d'AST, vérifiez que vous ne mêlez pas entiers et flottants dans vos opérations.}", ou "\inlinecode{Erreur de syntaxe, vérifiez que vous écrivez bien une expression arithmétique en notation infixe.}" selon le type d'erreur.
\subsection{Détails}
Le gros du travail se trouve dans le fichier \texttt{compiler.ml} renvoyant le \texttt{program} (cf. \texttt{x86\_64.ml}) associé à l'AST. En quelques mots, j'y évalue mon AST en émulant un DFS (en fait ici, un parcours infixe) de celui-ci : j'opère sur ses sous-arbres de façon récursive, puis j'effectue l'opération correspondant au constructeur de l'AST total avec comme arguments les (registres contenant les) résultats issus des appels récursifs sur les sous-arbres. Le cas de base est celui de la rencontre d'une feuille \texttt{Int(x)} ou \texttt{Float(x)}. Cette émulation se fait en pratique par l'utilisation de la pile pour pouvoir y stocker les résultats et les extraire pour agir dessus. Le fondement derrière est que le principe LIFO d'une pile est utilisé pour les DFS : on réutilise ici cette idée. Le résultat de cette compilation sert dans le fichier \texttt{main.ml} pour générer le programme \texttt{expression.s}. \par
Les opérations de l'énoncé ont l'air de fonctionner, je n'ai pas fait de bonus. J'ai implémenté les opérateurs unaires de début d'expression \inlinecode{-(exp)}, \inlinecode{-.(exp)}, \inlinecode{+(exp)} et \inlinecode{+.(exp)} pour simplifier la gestion des cas entre expressions d'entiers et de flottants, surtout pour le cas du passage à l'opposé où les instructions assembleur sont différentes. En revanche, dans le cas d'un signe positif avant une expression, j'évalue en fait seulement cette expression, donc en réalité on peut mettre au choix un \inlinecode{+} ou un \inlinecode{+.} en début d'expression parenthésée, ça ne sera pas source d'erreur de typage de l'AST. (Je ne sais pas s'il s'agit vraiment d'une négligence de ma part, je dirais juste que je garde des conventions d'écriture.) Différents tests sont disponibles dans le dossier \textsf{Tests}.\newline
\newline
\begin{center}
    *  *  *
\end{center}
\end{document}
